1. The answers to these questions should be answerable without referring
to external materials. \textbf{Please input brief explanation for T/F questions as well}.
\begin{enumerate}
    \item\points{2} In your own words, describe what bias and variance are. What is the bias-variance tradeoff?
	Prove that $\| x \| = f(x) = \left( \sum_{i=1}^n x_i^2 \right)^{\frac{1}{2}}$ is a norm.
\begin{itemize}
    \item Non-negativity: $\| x \| \geq 0$ for all $x_i \in \R^n$ since $x_i^2 \geq 0$ and $\| x \| = 0$ if and only if $x = 0$.
    \item Absolute scalability: $\| ax \| = \lvert a \rvert \| x \|$ for all $a \in \R$ and $x \in \R^n$. 
	\begin{align*}
	    \| ax \| &= \left( \sum_{i=1}^n (ax_i)^2 \right)^{\frac{1}{2}} \\
		&= \left( \sum_{i=1}^n a^2 x_i^2 \right)^{\frac{1}{2}} \\
		&= \left(a^2 \sum_{i=1}^n x_i^2 \right)^{\frac{1}{2}} \\
		&= \left( a^2 \right)^{\frac{1}{2}} \left( \sum_{i=1}^n x_i^2 \right)^{\frac{1}{2}} \\
		&= \lvert a \rvert \left( \sum_{i=1}^n x_i^2 \right)^{\frac{1}{2}} \tag*{since $\left( a^2 \right)^{\frac{1}{2}} > 0$ for all $a \in \R \setminus 0$} \\
		&= \lvert a \rvert \| x \|
	\end{align*}
    \item Triangle inequality: $\| x + y \| \leq \| x \| + \| y \|$ for all $x, \, y \in \R^n$. 
	Let $\langle x, y \rangle$ be the inner product $x \cdot y = \sum_{i=1}^n x_i y_i$. 
	Prove using Cauchy-Schwarz inequality, i.e. $\lvert \langle x, y \rangle \rvert^2 \leq \langle x, x \rangle \cdot \langle y, y \rangle$.
	\begin{align*}
	    \left( \| x + y \| \right)^2 &= \| x \|^2 + 2 \langle x, y \rangle + \| y \|^2 \\
		&\leq \| x \|^2 + 2 \left( \| x \| \cdot \| y \| \right) + \| y \|^2 \\
		&= \left( \| x \| + \| y \| \right)^2
	\end{align*}
	Thus $\| x + y \| \leq \| x \| + \| y \|$ since $\left( \| x + y \| \right)^2 = \left( \| x \| + \| y \| \right)^2$ and all are positive by non-negativity.
\end{itemize}
    
	
    \item \points{2} What happens to bias and variance when model complexity increases/decreases? 
	Prove by contradiction that $g(x) = \left( \sum_{i=1}^n \lvert x_i \lvert^{\frac{1}{3}} \right)^3$ is not a norm.
\begin{itemize}
    \item[] Assume that $g(x)$ is a norm. 
    \item[] By the triangle inequality $g(x + y) = g(x) + g(y)$ for all $x, y \in \R^n$.
    \item[] Let $x = [1, 0]^T$ and $y = [0, 1]^T$. 
    \item[] Trivially, $x, y \in \R^n$.
    \item[] $g(x + y) = (1 + 1)^3 = 8$
    \item[] $g(x) + g(y) = 1^3 + 1^3 = 2$
    \item[] $g(x + y) \neq g(x) + g(y)$
\end{itemize}
Thus, since $g(x)$ is a norm and the triangle inequality doesn't hold is a contradiction, $g(x)$ is not a norm.

    \item \points{1} True or False: The bias of a model increases as the amount of available training data increases.
	\begin{itemize}
   \item[] {\bf False} -- Typically more data allows learning better estimator parameters which reduce bias. 
\end{itemize} 


    \item \points{1} True or False: The variance of a model decreases as the amount of available training data increases.
	\begin{itemize}
   \item[] {\bf True} -- Typically more data reduces variance if the estimator is well chosen. However, if the estimator is poor, then this might not be the case. 
\end{itemize}

    \item \points{1} True or False: A learning algorithm will always generalize better if we use fewer features to represent our data.
	\begin{itemize}
    \item[] {\bf False} -- Too few features can cause our model to generalize worse on unseen data.
\end{itemize}


    \item \points{2} To obtain superior performance on new unseen data, should we use the training set or the test set to tune our hyperparameters?
	\begin{itemize}
     \item[] {\bf Neither} -- We should use validation data to tune hyperparameters. However, if for some reason this is not an option, then we should use training data to tune hyperparameters. {\bf Never} use test data to construct a model.
\end{itemize}

    \item \points{1} True or False: The training error of a function on the training set provides an overestimate of the true error of that function.
	\begin{itemize}
    \item[] {\bf False} -- Typically the model has lower training error on training data than the test error on test data since, although the model is tuned to training data with the assumption that test data is like training data, this isn't always the case and thus test error is likely higher than training error.
\end{itemize}


    \item \points{1} True or False: Using L2 regularization when training a linear regression model encourages it to use less input features when making a prediction.
	\begin{itemize}
    \item[] {\bf False} -- L2 Regularization reduces the impact of features on the model to reduce overfitting. It does not, however, actually remove or limit the number of features that the model has. 
\end{itemize}

\end{enumerate}
