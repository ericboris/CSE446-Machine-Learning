1. The answers to these questions should be answerable without referring
to external materials. \textbf{Please input brief explanation for T/F questions as well}.
\begin{enumerate}
    \item\points{2} In your own words, describe what bias and variance are. What is the bias-variance tradeoff?
	\begin{itemize}
    \item[] {\bf Bias} -- The measure of how closely our model matches the best possible estimator, with lower values representing a closer match.
    \item[] {\bf Variance} -- The measure of how much our model changes over i.i.d subsets of training data from the true probability distribution with lower values representing lower average differences.
    \item[] {\bf Bias-Variance Tradeoff} -- Together, bias and variance represent the two types of learning error. However, one cannot usually be reduced completely without increasing the other. This is because {\bf lowering bias tends to overfit the model which increases the variance}. Conversely, {\bf lowering variance reduces fit which increases bias}. Thus, an optimal model balances bias and variance accepting some degree of error.
\end{itemize}
	
    \item \points{2} What happens to bias and variance when model complexity increases/decreases? 
	\begin{itemize}
    \item[] {\bf Increased complexity} leads to {\bf decreased bias} and {\bf increased variance}. {\bf Decreased complexity} leads to {\bf increased bias} and {\bf decreased variance}.
\end{itemize}

    \item \points{1} True or False: The bias of a model increases as the amount of available training data increases.
	\begin{itemize}
   \item[] {\bf False} -- Typically more data allows learning better estimator parameters which reduce bias. 
\end{itemize} 


    \item \points{1} True or False: The variance of a model decreases as the amount of available training data increases.
	\begin{itemize}
   \item[] {\bf True} -- Typically more data reduces variance if the estimator is well chosen. However, if the estimator is poor, then this might not be the case. 
\end{itemize}

    \item \points{1} True or False: A learning algorithm will always generalize better if we use fewer features to represent our data.
	\begin{itemize}
    \item[] {\bf False} -- Too few features can cause our model to generalize worse on unseen data.
\end{itemize}


    \item \points{2} To obtain superior performance on new unseen data, should we use the training set or the test set to tune our hyperparameters?
	\begin{itemize}
     \item[] {\bf Neither} -- We should use validation data to tune hyperparameters. However, if for some reason this is not an option, then we should use training data to tune hyperparameters. {\bf Never} use test data to construct a model.
\end{itemize}

    \item \points{1} True or False: The training error of a function on the training set provides an overestimate of the true error of that function.
	\begin{itemize}
    \item[] {\bf False} -- Typically the model has lower training error on training data than the test error on test data since, although the model is tuned to training data with the assumption that test data is like training data, this isn't always the case and thus test error is likely higher than training error.
\end{itemize}


    \item \points{1} True or False: Using L2 regularization when training a linear regression model encourages it to use less input features when making a prediction.
	\begin{itemize}
    \item[] {\bf False} -- L2 Regularization reduces the impact of features on the model to reduce overfitting. It does not, however, actually remove or limit the number of features that the model has. 
\end{itemize}

\end{enumerate}
