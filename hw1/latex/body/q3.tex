3. You're a Reign FC fan, and the team is five games into its 2018 season. The
number of goals scored by the team in each game so far are given below:

\[
  [2, 0, 1, 1, 2].
\]
Let's call these scores $x_1, \dots, x_5$. Based on your (assumed iid)
data, you'd like to build a model to understand how many goals the
Reign are likely to score in their next game. You decide to model the
number of goals scored per game using a \emph{Poisson
  distribution}. The Poisson distribution with parameter $\lambda$
assigns every non-negative integer $x = 0, 1, 2, \dots$ a probability
given by
\[
  \mathrm{Poi}(x | \lambda) = e^{-\lambda} \frac{\lambda ^ x}{x!}.
\]
So, for example, if $\lambda = 1.5$, then the probability that the
Reign score 2 goals in their next game is
$e^{-1.5} \times \frac{1.5^2}{2!} \approx 0.25$. To check your
understanding of the Poisson, make sure you have a sense of whether
raising $\lambda$ will mean more goals in general, or fewer.

\begin{enumerate}
\item \points{5} Derive an expression for the maximum-likelihood
  estimate of the parameter $\lambda$ governing the Poisson
  distribution, in terms of your goal counts $x_1, \dots,
  x_5$. (Hint: remember that the log of the likelihood has the same
  maximum as the likelihood function itself.)
    \input{body/a3_a.tex}
    
\item \points{5} Suppose the team scores 4 goals in its sixth
  game. Derive the same expression for the estimate of the parameter
  $\lambda$ as in the prior example, now using the 6 games
  $x_1, \ldots, x_5, x_6 =4$.
    \begin{quote}
    When $n=6$ $$\boxed{ l_{{\rm MLE}} = \frac{1}{6} \sum_{i=1}^6 x_i }$$
\end{quote}


\item \points{5} Given the goal counts, please give  numerical
  estimates of $\lambda$ after 5 and 6 games.
    \begin{quote}
    $l_{{\rm MLE, 5}} = \frac{1}{5} \sum_{i=1}^5 x_i = \frac{1}{5} \sum [2, 0, 1, 1, 2] = \boxed{\frac{6}{5}}$ \\ 
    $l_{{\rm MLE, 6}} = \frac{1}{6} \sum_{i=1}^6 x_i = \frac{1}{6} \sum [2, 0, 1, 1, 2, 4] = \frac{10}{6} = \boxed{\frac{5}{3}}$
\end{quote}

\end{enumerate}
