\documentclass[11pt]{article}
\usepackage{amsmath, amsfonts, amsthm, amssymb}  % Some math symbols
\usepackage[utf8x]{inputenc}
\usepackage{fullpage}
\usepackage[x11names, rgb]{xcolor}
\usepackage{graphicx}
\usepackage{tikz}
\usepackage{etoolbox}
\usepackage{enumerate}
\usepackage{enumitem}
\usepackage{listings}
\usepackage{hyperref}
\usepackage{lipsum}
\usepackage{sectsty}
\usepackage{verbatim}
\usetikzlibrary{decorations,arrows,shapes}

%% Define the title contents
\title{}
\author{}
\date{}

%% Set the size of the section header
\sectionfont{\fontsize{11}{12}\selectfont}

%% Set the size and format of the subsection header
\subsectionfont{\fontsize{11}{12}\selectfont}
\renewcommand{\thesubsection}{\thesection (\alph{subsection})}

%% Set the size and format of the subsubsection header
\subsubsectionfont{\fontsize{9}{10}\selectfont}
\renewcommand{\thesubsubsection}{\roman{subsubsection}}

%% Define Real and Rational numbers symbol
\newcommand{\R}{\mathbb{R}}
\newcommand{\Q}{\mathbb{Q}}
\newcommand{\N}{\mathbb{N}}
\newcommand{\Z}{\mathbb{Z}}

%% Redefine rightarrow to imp
\def\imp{\rightarrow}

%% Redefine overline to ol
\def\ol{\overline}

%% Redefine leftrightarrow to lra
\def\lra{\leftrightarrow}

% Redefine setminus to sm
\def\sm{\setminus}

%% Define a nested environment using level for formal proof
\newenvironment{level}%
{\addtolength{\itemindent}{2em}}%
{\addtolength{\itemindent}{-2em}}



%% Set enumerate sub list to use numbers instead of letters
\setlist[enumerate]{label*=\arabic*.}

%% Define custom style
\lstdefinestyle{myCustomMatlabStyle}{
  language=Java,
  numbersep=10pt,
  tabsize=4,
  showspaces=false,
  showstringspaces=false
}

%% Define the default code language to Java
\lstset{basicstyle=\small, style=myCustomMatlabStyle}

%%--- Begin the Document ---%%

\begin{document}

\section*{P1}

\noindent\textcolor[RGB]{220,220,220}{\rule{\linewidth}{0.8pt}}

\subsection*{Claim:} 
A connected graph G with m $\geq$ n edges can have it's edges oriented such that the degree of every vertex is at least 1. We use DFS to find a vertex u that is part of a cycle. We use DFS again starting from u. To every pair of vertices \{v, w\} encountered after u we assign directed outgoing edges (v, w). A special case is made for u on the first loop iteration. Only the first vertex x that u encounteres receives a directed incoming edge (u, x). All other edges from u on the first iteration are left undirected.

\subsection*{Algorithm:} 
\begin{lstlisting}[basicstyle=\small]
// Where G is a connected graph with m >= n edges
OneIncoming(G):
	// Use modifiedDFS from HW2 P3a to find the vetices of a cycle in G
	Set V to modifiedDFS(G)
	Set v such that v is the first vertex in V
	// Output the orientation of edges of G
	// such that every vertex has at least one incoming edge
	Orient(v):
		Initialize S to be a stack with one element s
		Set b to true representing the first iteration
		Set L to be an empty list of directed edges
		While S is not empty
			Take a node u from S
			If Explored[u] = false then
				Set Explored[u] = true
				// Every iteration but the first
				If b is false then 
					For each edge (u, v) incident to u
						Add v to the stack S
						// All new edges are outgoing
						If directed edge (v, u) not in L 
							Add directed edge (u, v) to L
						EndIf
					EndFor
				// Handle the first iteration differently
				Else if b is true then
					For each edge (u, v) incident to u
						Add v to the stack S
						// Only the first edge is outgoing
						If b is true then 
							Set b to false
							Add directed edge (u, v) to L
						EndIf
					EndFor
				EndIf
			EndIf
		EndWhile
		Return L
\end{lstlisting}

\noindent\textcolor[RGB]{220,220,220}{\rule{\linewidth}{0.8pt}}
\linebreak

\subsection*{Claim:} 
The algorithm terminates in O(m + n) time. 

\subsection*{Proof:}
We are using two instances of DFS which is known to terminate in O(m + n). Because no modification has added additional runtime to DFS beyond O(1) work, both instances still terminate in O(m + n). Because one instance of DFS is run after another, the runtime of the sum of the instances of DFS is O(m + n). Thus, the overall runtime of the algorithm is O(m + n). $\qed$

\noindent\textcolor[RGB]{220,220,220}{\rule{\linewidth}{0.8pt}}

\subsection*{Claim:} 
The algorithm orients the edges of connected graph G with m $\geq$ n edges such that that indegree of every vertex is at least 1.

\subsection*{Proof:}
Consider arbitrary graph G with n vertices and m $\geq$ n edges. 
We know from HW2 P3 that any graph with m $\geq$ n edges contains a cycle. So, G contains a cycle. We also know from HW2 P3 that any graph with a cycle has at least one edge that can be removed from that cycle without disconnecting the graph. Remove such an edge e between vertices \{u, v\} from G to create graph G' where G' = G - e. We know DFS can reach every connected vertex in G' starting from u. G' is connected so we can reach every vertex in G' from u. Use DFS starting from u. For every edge between vertices \{x, y\} in G' encountered by DFS, draw a directed edge (x, y). Because G' is connected we know v will be encountered by DFS from another vertex z with an incoming edge (z, v). We now know that all vertices in G' except for u have an incoming edge. Copy the direction of every edge in G' to the corresponding edge in G such that all vertices in G have at least one incoming edge except for u. Add the directed edge (v, u) to G. Thus, all vertices in G have at least one incoming edge. $\qed$

\end{document}